\documentclass{article}
\usepackage[utf8]{inputenc}
\usepackage{graphicx}
\usepackage{amsmath}
\usepackage{siunitx}

\usepackage{hyperref}
\usepackage[style=numeric-comp,sorting=none,hyperref=true,backend=biber]{biblatex}

\usepackage{cleveref}

\begin{document}
	
	\subsubsection{Pipe model V2}
	
	Surface resistance:
	
	\begin{equation}
		R_f = f_m \cdot \frac{8 \cdot L \cdot \rho}{\pi^2 \cdot D^5}
	\end{equation}
	
	Form resistance
	
	\begin{equation}
		R_f = k_f \cdot \frac{8 \cdot \rho}{\pi^2 \cdot D^4}
	\end{equation}
	
	
	\begin{equation}
		\lambda (\dot q) = (R_f + R_m) \cdot |q| \cdot q
	\end{equation}


\subsubsection{Valve model}

The relationship between pressure difference and flow is the same for all flows and pressure differences:
\begin{equation}
	\frac{\Delta p_1}{q_1^2} = \frac{\Delta p_2}{q_2^2}
\end{equation}

Isolating one flow:
\begin{equation}
	q_1 = q_2 \cdot \sqrt{\frac{\Delta p_1}{\Delta p_2}}
\end{equation}

There will be a flow $q_2$ where the pressure drop will be equal to one and thus:
\begin{equation}
	q_1 = q_2 \cdot \sqrt{\Delta p_1}
\end{equation}

Writing this up for all flows
\begin{equation}
	q = k_v(OD) \cdot \sqrt{\Delta p}
\end{equation}

Isolating the pressure difference:
\begin{equation}
	\Delta p = \frac{1}{k_v(OD)} \cdot q^2 = \frac{1}{k_v(OD)} \cdot |q| \cdot q
\end{equation}

\begin{center}
	\begin{tabular}{l p{8cm} l}
		$h_v(OD)$ & Constant dependent on valve opening degree (OD) \\
	\end{tabular}
\end{center}


\subsubsection{Tank model}
The pressure in the bottom of the tank is proportional to the level of the tank:
\begin{equation}
	P_{v4} \propto h_{tank}
\end{equation}

\begin{center}
	\begin{tabular}{l p{8cm} l}
		$h_{tank}$ & is the level in the tank & [\si{m}]\\
	\end{tabular}
\end{center}

The following is true if the tank cross-sectional area (A) is constant
\begin{equation}
	\dot V_{\tau} = d_{\tau}
\end{equation}

As such the follow is true
\begin{equation}
		\dot V_{\tau} \propto h_{tank} \propto d_{\tau}
\end{equation}

And thus
\begin{equation}
	\dot p_{v4} \propto d_t
\end{equation}

Which leads us to
\begin{equation}
	\dot p_{v4} = -\tau \cdot d_t
\end{equation}


\subsubsection{Pump model}

The full pump model:
\begin{equation}
	\Delta p = a_2 \cdot q^2 - a_1 \cdot q - a_0 \cdot q \cdot \omega^2
\end{equation}

\begin{center}
	\begin{tabular}{l p{8cm} l}
		$a_0 - a_2$ & are constants obtained from measurements of the pump \\
	\end{tabular}
\end{center}

We disregard the ??? and thus end up with:
\begin{equation}
	\Delta p = a_2 \cdot q^2 - a_0 \cdot q \cdot \omega^2
\end{equation}

\end{document}
