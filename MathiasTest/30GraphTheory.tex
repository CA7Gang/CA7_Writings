\section{System modelling }
This section describes the interconnected component in a water distributed network (WDN) using Graph Theory. Furthermore thee component classes will be examined, ie. pipes, pumps and valves.


\subsection{Graph Theory}
When using graph theory as an analytical tool the incident matrix $H$ comes in handy when describing the connection between edges and nodes. The rules shown below has been used as guidelines when writing up the incident matrix for a graph network. 

\subsubsection{The Incident Matrix}
\begin{equation*}
H_{i,j} = \begin{cases}
    -1 & \text{If the $j^{th}$ edge enters the $i^{th}$ node} \\
    0 & \text{If the $j^{th}$ edge is not connected to the $i^{th}$ node} \\
    1 & \text{If the $j^{th}$ edge s leaving the $i^{th}$ node}
\end{cases}
\end{equation*} %Description of the incident matrix

\begin{figure}[h!]
	\centering
	\includegraphics[width=0.5\textwidth]{Pictures/Graph.png}
	\caption{Graph of simplified WDN network \cite{Rathore930}}
	\label{fig:graph}
\end{figure}

When applying the rules shown above for the simplified graph model of the WDN the incident matrix in \cref{eq:H_simplified}.
\begin{equation}
    H = \begin{bmatrix}
1 & 0 & 1 & 0 & 0 & 0 & 0\\
-1 & 1 & 0 & 0 & 0 & 0 & 0\\
0 & 0 & -1 & 1 & 1 & 0 & 0\\
0 & -1 & 0 & -1 & 0 & 1 & 0\\
0 & 0 & 0 & 0 & -1 &  0  & -1\\
0 & 0 & 0 & 0 & 0 & -1 & 1
\end{bmatrix}
\label{eq:H_simplified}
\end{equation} %The incident matrix for system

The reduced incident matrix by taking an arbitrary vertex as a reference, and removing that vertex-row from \cref{eq:H_simplified}. We chose the $4^{th}$ vertex, which results in the following reduced incident matrix:
\begin{equation}
    \bar{H} = \begin{bmatrix}
1 & 0 & 1 & 0 & 0 & 0 & 0\\
-1 & 1 & 0 & 0 & 0 & 0 & 0\\
0 & 0 & -1 & 1 & 1 & 0 & 0\\
0 & 0 & 0 & 0 & -1 &  0  & -1\\
0 & 0 & 0 & 0 & 0 & -1 & 1
\end{bmatrix}
\end{equation}

Chords and edges of the spanning tree

\begin{equation*} 
	\begin{split}
E_{C} &= \{e_{1},e_{4}\}   \\ E_{T} &= \{e_2,e_3,e_5,e_6,e_7\}
	\end{split}
\end{equation*}


\subsubsection{The Loop Marix}
The loop-matrix can be calculated in one of two ways. The first one is shown in the equation below

\begin{equation}
    B = \begin{bmatrix}
I & -\bar{H}_{C}^{T}\cdot\bar{H}_{T}^{-T}
\end{bmatrix}
\end{equation}

With $H_{c}$ and $H_{T}$ being the cord and tree component of the reduced incident matrix. The result of loop matrix is shown below.

\begin{equation}
	B = \begin{bmatrix}
		1 & 0 & 1 & -1 & -1 & 1 & 1\\
		0 & 1 & 0 & 0 & -1 & 1 & 1\\
		\end{bmatrix}
\end{equation}

The other method is more graphical but can be formulated as follows: JEG KAN IKKE HUSKE REGLEN.