\section{Component model}
All components (eg. pipes, valves and pumps) can be described by two variables, namely the flow through the component and the differential pressure across the component:
\begin{equation}
\begin{bmatrix} \Delta{p_{k}} \\ q_{k} \end{bmatrix} = 
\begin{bmatrix} p_{i} - p_{j} \\ q_{k} \end{bmatrix}    
\end{equation}

The following section will examine these two variables for pipes, valves and pumps.

\subsection{Pipe model}
The differential pressure across a pipe can be modelled as follows:
\begin{equation}
    \mathrm{\Delta{p_{k}}} = J_{k}\cdot\dot{q_{k}}+\mathrm{\lambda_{k}}(q_{k})-\mathrm{\Delta{z_{k}}}
\end{equation}


	\begin{center}
		\begin{tabular}{l p{8cm} l}
			
			$\Delta{p_{k}}$ & The differential pressure across the $k^{th}$ component & [\si{Pa}]\\ 
		  	${J_{k}}$ & Is the mass inertia of the water in the $k^{th}$ pipe & [\si{kg}/\si{m^{4}}] \\
		  	$q_{k}$ & is the flow of water trough the $k_{th}$ pipe & [{\si{\meter\cubed}/\si{s}}] \\
		  	$\mathrm{\lambda_{k}}(q_{k})$ & is the drop in pressure due to friction in the $k^{th}$ pipe & [\si{Pa}] \\
		  	$\mathrm{\Delta{z_{k}}}$ & is the drop in pressure due to geodesic level & [\si{Pa}]\\
			\end{tabular}
	\end{center}

The mass inertia of water can be describes as follows:
\begin{equation}
	J= \frac{L\cdot \rho}{A}
\end{equation}

	\begin{center}
		\begin{tabular}{l p{8cm} l}
			$L$ & is the length of the pipe & [\si{m}]\\
			$\rho$ & is the density of water & [\si{kg}/\si{m\cubed}]\\  
			$A$ & is the cross sectional area of water & [\si{m\squared}]\\ 
		\end{tabular}
	\end{center}
The cross-sectional area of the pipe is assumed to be constant along the pipe.


\subsubsection{Causes of friction}
The causes of flow friction $\mathrm{\lambda_{k}}(q_{k})$ are surface resistance $h_{f}$ and form resistance $h_{m}$. The surface resistance can be describes with the Darcy-Weisbach equation:

\begin{equation}
	h_{f} = f \cdot \frac{8\cdot L\cdot q^{2}}{\pi^{2}\cdot g \cdot D^{5}}
\end{equation} 

\begin{center}
	\begin{tabular}{l p{8cm} l}
		$h_{f}$ & is the total loss from surface resistance & [\si{m}]\\
		$f$ & is the pipe friction factor & [$\cdot$]\\
		$D$ & is the pipe diameter & [\si{m}]\\
		$g$ & is the gravitational constant & [\si{m}/\si{s\squared}]\\
	\end{tabular}
\end{center}
Under the assumption of turbulent flow $f$ can be is given by:
\begin{equation}
	f=1.325\cdot \Bigg(ln\Big(\frac{\epsilon}{3.7 \cdot D}+\frac{5.74}{R^{0.9}}\Big)\Bigg)^{-2}
\end{equation}

\begin{center}
	\begin{tabular}{l p{8cm} l}
		$\epsilon$ & average height of roughness projection in the pipe & [\si{m}]\\
		$R$ &  is Reynolds number - for turbulent flow $R \geq 4000$ & [$\cdot$]\\
	\end{tabular}
\end{center}

The form resistance can be given by the following:
\begin{equation}
	h_{m}=k_{f}\cdot \frac{8\cdot q^{2}}{\pi^{2}\cdot g \cdot D^{4}}
\end{equation}

\subsection{Valve model}
Something about the valve model

\subsection{Pump model}
Something about the pump model
