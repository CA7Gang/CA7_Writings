\documentclass[10pt]{beamer}
\usetheme[
%%% options passed to the outer theme
%    hidetitle,           % hide the (short) title in the sidebar
%    hideauthor,          % hide the (short) author in the sidebar
%    hideinstitute,       % hide the (short) institute in the bottom of the sidebar
%    shownavsym,          % show the navigation symbols
%    width=2cm,           % width of the sidebar (default is 2 cm)
%    hideothersubsections,% hide all subsections but the subsections in the current section
%    hideallsubsections,  % hide all subsections
    left               % right of left position of sidebar (default is right)
%%% options passed to the color theme
%    lightheaderbg,       % use a light header background
  ]{AAUsidebar}

% If you want to change the colors of the various elements in the theme, edit and uncomment the following lines
% Change the bar and sidebar colors:
%\setbeamercolor{AAUsidebar}{fg=red!20,bg=red}
%\setbeamercolor{sidebar}{bg=red!20}
% Change the color of the structural elements:
%\setbeamercolor{structure}{fg=red}
% Change the frame title text color:
%\setbeamercolor{frametitle}{fg=blue}
% Change the normal text color background:
%\setbeamercolor{normal text}{bg=gray!10}
% ... and you can of course change a lot more - see the beamer user manual.


\usepackage[utf8]{inputenc}
\usepackage[english]{babel}
\usepackage[T1]{fontenc}
% Or whatever. Note that the encoding and the font should match. If T1
% does not look nice, try deleting the line with the fontenc.
\usepackage{helvet}
\usepackage{tikz}
\usetikzlibrary{shapes,shapes.geometric, arrows,positioning,calc}
\tikzset{
	block/.style = {draw, fill=white, rectangle, minimum height=3em, minimum width=3em},
	tmp/.style  = {coordinate}, 
	sum/.style= {draw, fill=white, circle, node distance=1cm},
	input/.style = {coordinate},
	output/.style= {coordinate},
	pinstyle/.style = {pin edge={to-,thin,black}
	}
}

% colored hyperlinks
\newcommand{\chref}[2]{%
  \href{#1}{{\usebeamercolor[bg]{AAUsidebar}#2}}%
}

\title[Modelling and Networked Control of Water Distribution Networks]% optional, use only with long paper titles
{Modelling and Networked Control of Water Distribution Networks}

% \subtitle{}  % could also be a conference name

\date{\today}

\author[CA733] % optional, use only with lots of authors
{
  CA733
}
% - Give the names in the same order as they appear in the paper.
% - Use the \inst{?} command only if the authors have different
%   affiliation. See the beamer manual for an example

\institute[
%  {\includegraphics[scale=0.2]{aau_segl}}\\ %insert a company, department or university logo
  Control and Automation\\
  Aalborg University\\
  Denmark
] % optional - is placed in the bottom of the sidebar on every slide
{% is placed on the title page
  Control and Automation, Group 733\\
  Aalborg University\\
  Denmark
  
  %there must be an empty line above this line - otherwise some unwanted space is added between the university and the country (I do not know why;( )
}


% specify a logo on the titlepage (you can specify additional logos an include them in 
% institute command below
\pgfdeclareimage[height=1.5cm]{titlepagelogo}{AAUgraphics/aau_logo_new} % placed on the title page
%\pgfdeclareimage[height=1.5cm]{titlepagelogo2}{graphics/aau_logo_new} % placed on the title page
\titlegraphic{% is placed on the bottom of the title page
  \pgfuseimage{titlepagelogo}
%  \hspace{1cm}\pgfuseimage{titlepagelogo2}
}


\begin{document}
% the titlepage
{\aauwavesbg%
\begin{frame}[plain,noframenumbering] % the plain option removes the sidebar and header from the title page
  \titlepage
\end{frame}}
%%%%%%%%%%%%%%%%

% TOC
\begin{frame}{Agenda}{}
\tableofcontents
\end{frame}
%%%%%%%%%%%%%%%%

\section{Introduction}


% motivation for creating this theme
\begin{frame}{Introduction}{Introduction to Water Distribution Networks}
\begin{itemize}
	\item Critical societal infrastructure, responsible for the provision of water to both domestic and industrial consumers. 
	\item Network pressure must be controlled.
	\begin{itemize}
		\item Underpressure $\rightarrow$ insufficient service pressure $\rightarrow$ dissatisfied end users.
		\item Overpressure $\rightarrow$ component failure $\rightarrow$ repair costs, supply intermittency, etc.
	\end{itemize}
	\item Network pressure tied to level in Elevated Water Reservoir(s) (EWR) $\rightarrow$ level control is key!
	\item Other key components are pumps, valves, and pipes.
\end{itemize}
\end{frame}
\begin{frame}{Introduction}{Test Water Distribution Network Layout}
\begin{block}{}
	We analyse a small-scale test WDN with the following layout:
\end{block}

\begin{figure}[h]
	\centering
	\includegraphics[width=0.5\linewidth]{Graphics/WDNModel.pdf}
	\label{fig:WDNModel}
	\begin{block}{}
		Larger WDNs are typically not amenable to first-principles modelling.
	\end{block}
\end{figure}

\end{frame}
%%%%%%%%%%%%%%%%

% ======================================================================
% Inputs for topics here!

\section{Modelling of WDN}

\begin{frame}{Modelling of WDN}
	divided in two sections..?
\end{frame}
\section{Fast Dynamics and Graph theory}
%\input{Topics/XX/XX}
%
\section{Slow Dynamics and Linearisation}
\subsubsection{Slow Dynamics}

\begin{frame}{Modelling of Water Distribution Networks}{Slow Dynamics}
	\begin{columns}
		\begin{column}{.5\textwidth}
			\begin{itemize}
				\item Fundamentals
				\begin{equation*}
					p \propto h 
				\end{equation*}
				\begin{equation*}
					\dot{V} = q
				\end{equation*}
				\item Assume constant cross sectional area A
				\begin{equation*}
					V \propto h \implies V \propto p 
				\end{equation*}
				\begin{equation*}
					\dot{p} \propto \dot{V} \wedge \dot{V} = q \implies \dot{p} \propto q
				\end{equation*}
				\item We arrive at
				\begin{equation*}
					\dot{p} = -\tau q \text{,  where}
				\end{equation*}	
				\begin{equation*}
					\tau = \rho g \frac{1}{A}
				\end{equation*}	
			\end{itemize}
		\end{column}
		\begin{column}{.5\textwidth}\raggedleft
			\includegraphics[width=1\linewidth]{Topics/SlowDynamicsLinearisation/Graphics/Tank_sketch.png}
		\end{column}
	\end{columns}
\end{frame}

\begin{frame}{Modelling of Water Distribution Network}{State-space Formulation of Slow Dynamics}
	\begin{figure}[h!]
		\centering
		\resizebox{\columnwidth}{!}{
			\input{Topics/SlowDynamicsLinearisation/TikzFigures/TikzGraphNetwork.tex}}
		\label{fig:tikzWDNGraph}
	\end{figure}  
	In context of WDN we now consider flows to and from network as external demands $ d_i $
\end{frame}

\begin{frame}{Modelling of Water Distribution Networks}{State-space Formulation of Slow Dynamics}
	Mass conservation holds, and such
	\begin{equation*}
		d_n = -\sum_{i=1}^{n-1}d_i \implies d_\tau = - (d_p + d_c)
	\end{equation*}
	\begin{equation*}
		\dot{p} = -\tau d_\tau = \tau (d_p + d_c)
	\end{equation*}	
	When discretised by forward Euler:
	\begin{equation*}
		p_\tau(k+1) = p_\tau(k) - \tau d_\tau(k) t_s = p_\tau(k) + \tau(d_p(k) + d_c(k)) t_s
	\end{equation*}
	Which corresponds to a discrete, linear state-space model:
	\begin{equation}
		p_\tau(k+1) = Ap_\tau(k) + B_pd_p(k) + B_cd_c(k)
	\end{equation}
	\begin{equation*}
		d_p = \begin{bmatrix}
			d_1 \\ d_{13}
		\end{bmatrix},
		d_c = \begin{bmatrix}
			d_5 \\ d_9
		\end{bmatrix},
		B_p = B_c = t_s  \begin{bmatrix}
			\tau & \tau
		\end{bmatrix},
		A = 1
	\end{equation*}
\end{frame}


\subsubsection{Linearisation}
\begin{frame}{Modelling of Water Distribution Networks}{Linearisation}
	\begin{columns}
		\begin{column}{.4\textwidth}
			\begin{itemize}
				\item Fast dynamics are non-linear
				\item Linearisation required
			\end{itemize}
			In near vicinity of linearisation point $ x_0 $,
			\begin{equation*}
				\dot{x} \approx f(x_0) + \nabla f\bigg\rvert_{x_0} (x-x_0)
			\end{equation*}
			Linearising around equilibrium point preferred
		\end{column}
		\begin{column}{.6\textwidth}\raggedleft
			\includegraphics[width=1\linewidth]{Topics/SlowDynamicsLinearisation/Graphics/Linearisation_pic.png}
		\end{column}
	\end{columns}
\end{frame}

\begin{frame}{Modelling of Water Distribution Networks}{Linearisation}
%	Fast dynamics are non linear - linearisation is needed.\\
%	In near vicinity of linearisation point $ x_0 $,
%	\begin{equation*}
%		\dot{x} \approx f(x_0) + \nabla f\bigg\rvert_{x_0} (x-x_0)
%	\end{equation*}
	Recalling the fast dynamics differential equation is given as
	\begin{equation}\label{eq:NonLinearModelSimplified}
		\begin{split}
			\dot{q}_n &=  -\mathcal{P}\Phi\Big(\lambda(q_n)+\mu(q_n,\Theta)+\alpha(q_n,\omega)\Big) +\\ &\mathcal{P}\Big(\Psi(\bar{h}-\mathbf{1}h_0) + \mathcal{I}(p_{\tau}-\mathbf{1}p_0)\Big) \\
		\end{split}	
	\end{equation}
	The linear model such becomes
	\begin{equation}\label{eq:SymbolicLinearisation}
		\begin{split}
			\dot{q}_n &\approx f(x_0) + \frac{\partial f}{\partial q_n}\bigg\rvert_{x_0} \tilde{q}_n + \frac{\partial f}{\partial \Theta}\bigg\rvert_{x_0} \tilde{\Theta} + \frac{\partial f}{\partial \omega}\bigg\rvert_{x_0} \tilde{\omega} +  \frac{\partial f}{\partial p_\tau}\bigg\rvert_{x_0} \tilde{p}_\tau
			\\
		\end{split}
	\end{equation}
	where $x_0 = \{q_0,\Theta_0,\omega_0, p_{\tau_0} \}$, $ \tilde{q} =q-q_0$, likewise for $ \tilde{\Theta} $, $ \tilde{\omega}$, $\tilde{p_{\tau}}  $
\end{frame}

\begin{frame}{Modelling of Water Distribution Networks}{Linearisation}
	The full linearised model is then obtained as
	\begin{equation}\label{eq:SymbolicLinearisationExpanded}
		\begin{split}
			\dot{q}_n \approx f(x_0) -\mathcal{P}\Phi & \Bigg(a_1\omega_0 + \Big(|q_0|+\text{sign}(q_0)q_0\Big)\\
			& \Bigg(K_\lambda + a_2 + \frac{1}{(K_v \Theta_0)^2}\Bigg) \tilde{q}_n \Bigg)  \\
			- \mathcal{P}\Phi&\Bigg(\Big(-|q_0|q_0 \frac{2}{K_v^2 \Theta_0^3}\Big) \tilde{\Theta}\Bigg) \\
			- \mathcal{P}\Phi&\Bigg(\Big(a_1 q_0 + 2a_0\omega_0\Big) \tilde{\omega}\Bigg) \\
			+ \mathcal{P} \mathcal{I}& \tilde{p}_\tau
		\end{split}
	\end{equation}
The equation can be simplified further making some assumptions
\end{frame}

\begin{frame}{Modelling of Water Distribution Networks}{Linearisation}
	Equilibrium disappears, valve and tank dynamics assumed to be constant disturbances 
	\begin{equation}\label{eq:SymbolicLinearisationSimplified}
		\begin{split}
			\dot{q}_n \approx -\mathcal{P}\Phi &\Bigg(a_1\omega_0 + \Big(|q_0|+\text{sign}(q_0)q_0\Big) \\
			&\Bigg(K_\lambda + a_2 + \frac{1}{(K_v \Theta_0)^2}\Bigg) \tilde{q}_n \Bigg) \\
			- \mathcal{P}\Phi&\Bigg(\Big(a_1 q_0 + 2a_0\omega_0\Big) \tilde{\omega}\Bigg)
		\end{split}
	\end{equation}
	
\end{frame}

%
%\section{Control structure and root locus}
%\input{Topics/XX/XX}

\section{Optimal Control}

% motivation for creating this theme
\begin{frame}{Optimal Control}{General optimal control problem}
	The basic structure of an optimal control problem is sketched in the language of calculus of variations.
	\begin{align}
		\dot{x} = &f(x,u,t) \label{eq:BasicOptimalFunction} \\ 
		x(t_0) = x_0, \ &x \in \mathbb{R}^n, \ u \in U \in \mathbb{R}^m \label{eq:BasicOptimalDefinitions}
	\end{align}
	
	where $t \in \mathbb{R}$ is the time and $x,u$ are functions of $t$, with $U$ the set of admissible controls. \\
	\medskip
	Cost functional:
	\begin{equation}\label{eq:BolzaProblem}
		J = \mathcal{M}(x(T)) + \int_{0}^{T} \mathcal{L}(x(t),u(t)) dt
	\end{equation} 
\end{frame}
%%%%%%%%%%%%%%%%

%\begin{frame}{Linear-Quadratic Regulator}
% Let the dynamics of the (timevarying) system be given by:
%	\begin{equation}
%		\dot{x}(t) = A(t)x(t) + B(t)u(t) \label{eq:TimeVaryingLinearSystem}
%	\end{equation}
%
%	Cost functional:
%	\begin{equation}
%		J = \int_{t_0}^{t_1} \mathcal{L}(x(t),u(t)) dt + x^T(t_1)\mathcal{M}x(t_1)
%	\end{equation}
%	
%	Lagrangian:
%	\begin{equation}\label{eq:LQRLagrangian}
%		\mathcal{L}(x(t),u(t)) = x^T(t)Q(t)x(t) + u^T(t)R(t)u(t)
%	\end{equation}
%	
%	\begin{itemize}
%		\item Not time invariant!
%	\end{itemize}
%	
%\end{frame}

	%%%%%%%%%%%%%%%%
\begin{frame}{Infinite-Horizon Linear-Quadratic Regulator}
	The LQR timeinvariant case with no terminal cost.\\
	System Dynamics:
	\begin{equation}
		\dot{x}(t) = Ax(t) + Bu(t) \label{eq:TimeInvariantLinearSystem}
	\end{equation}

	Cost functional:		
	\begin{equation}\label{eq:LagrangeProblem}
		J = \int_{t_0}^{\infty} \big(x^T(t)Qx(t) + u^T(t)Ru(t)\big)dt
	\end{equation} 

	State feedback control law:
	\begin{equation}\label{eq:InfLQRFeedbackLaw}
		u^*(t) = -R^{-1}B^TPx^*(t)
	\end{equation}
	
	$P$ is time-invariant and fullfills the \textit{algebraic Riccati equation}:
	
	\begin{equation}\label{eq:ARE}
		PA + A^TP + Q - PBR^{-1}B^TP = 0
	\end{equation}
\end{frame}

	%%%%%%%%%%%%%%%%
\begin{frame}{Tracking LQR and Integral Action}{Tracking LQR}
	 
	
	Let $\hat{x} = x-x_r$ and $\hat{u} = u-u_r$. Shifted coordinate system cost functional framed as an output tracking problem:
	
	CHANGE NOTATION HERE
	
	
	\begin{equation}\label{eq:LagrangeProblemOutput}
		J = \int_{t_0}^{\infty} \big((C\hat{x})^TQ_y(C\hat{x}) + \hat{u}^TR\hat{u}\big)dt = \int_{t_0}^{\infty} \big(\hat{y}^TQ_y\hat{y} + \hat{u}^TR\hat{u}\big)dt
	\end{equation} 
	
	\begin{itemize}
		\item If linearized around some equlibrium point, these can be used as reference.
	\end{itemize}

\end{frame}


\begin{frame}{Tracking LQR and Integral Action}{Integral Action}
	States are extended with integral state $x_i$
	\begin{align}\label{eq:ClassicalIntegralAction}
		&u = -\bar{K}\bar{x} \\
		&\dot{\bar{x}} = \bar{A}\bar{x} + \bar{B}u + B_r r \\ 
		&y = \bar{C}\bar{x}\\
		&\bar{A} = \begin{bmatrix}A & 0 \\ -C & 0 \end{bmatrix}, \ \bar{B} = \begin{bmatrix} B \\ 0 \end{bmatrix}, \ B_r = \begin{bmatrix} 0 \\ 1 \end{bmatrix}, \ \bar{C} = \begin{bmatrix} C & 0 \end{bmatrix}\\
		& \bar{K} = -\begin{bmatrix} K & -K_i \end{bmatrix}
	\end{align}

	$K_i$ can be an awkward weight to choose.
\end{frame}

	%%%%%%%%%%%%%%%%
\begin{frame}{Velocity-Form LQR}
	Deviation variables:
	\begin{equation}\label{eq:VelocityVariables}
		\Delta x_k = x_k - x_{k-1}, \ \Delta y_k = y_k - r_k, \ \Delta u_k = u_k-u_{k-1}
	\end{equation}
	
	Extended vectors and matrices:
	\begin{equation} \label{eq1}
		\begin{split}
			& \tilde{\zeta}_k = \begin{bmatrix} \Delta x_k \\ \Delta y_k	\end{bmatrix}, \ \tilde{u}_k = \Delta u_k, \\
			&\tilde{A} = \begin{bmatrix} A & 0 \\ CA & I	\end{bmatrix}, \ 
			\tilde{B} = \begin{bmatrix} B \\ CB	\end{bmatrix}, \ \tilde{C} = \begin{bmatrix} 0 & I	\end{bmatrix}
		\end{split}
	\end{equation}

	Cost functional:
	\begin{equation}\label{eq:LagrangeProblemDeviation}
		J = \sum_{k_0}^{\infty} \big(\tilde{\zeta}^TQ\tilde{\zeta} + \tilde{u}^TR\tilde{u}\big)
	\end{equation}

	Origin regulation! If $\tilde{\zeta} \rightarrow 0 \Rightarrow  \Delta y \rightarrow 0 \Rightarrow  y \rightarrow r$.
\end{frame}

%Velocity-form dynamics:
%\begin{align}\label{eq:VelocityDynamics}
%	&\tilde{\zeta}_{k+1} = \tilde{A} \tilde{\zeta}_{k}  + \tilde{B}\tilde{u}_k \\
%	&\Delta y_k = \tilde{C}\tilde{x}_k
%\end{align}

\begin{frame}{Velocity-Form LQR}{Feedback Law}
	Control input applied at time k is:
	\begin{equation}\label{eq:ActualControlApplied}
		u^*(k) = \sum_{i=1}^{k} \Delta u^*(i)
	\end{equation}

	with
	\begin{equation}
		\tilde{K} = (\tilde{B}^TP\tilde{B}-R)^{-1}(\tilde{B}^TP\tilde{A})
	\end{equation}

	\begin{itemize}
		\item With low sampling frequency the correlation between decreases!
	\end{itemize}
\end{frame}


\begin{frame}{Exogenous input disturbance-accommodating LQR}

	Standard LQR does not accommodate exogenous inputs (such as the the model of the consumer demand flows) but can be modified to do so:
	\begin{equation}
		u(k) = \sum_{i=1}^{k} \Delta{u}^*(i) - B^\dagger \mathcal{B}\delta(i)
	\end{equation}
	
	where $B^\dagger$ is the Moore-Penrose pseudoinverse of $B$ and $\mathcal{B}$ is the disturbance input matrix.
\end{frame}


\section{Disturbance Estimator}
\begin{frame}{Disturbance Estimation}
	 \textbf{Objectives with the use of disturbance estimation}
	 \begin{itemize}
	 	\item VF-LQR controller needs an estimate of the consumer demand.
	 	\item The optimal estimator is the \textbf{Kalman Filter}.
	 \end{itemize}
\end{frame}

%%%%%%%%%%%%%%%%%

\begin{frame}{Disturbance Estimation}{The Kalman Filter}
 	\begin{itemize}
 		\item Need a linear model of consumer behaviour for the Kalman Filter
 		\item Under normal circumstances the Kalman gain is found recursively.
 		\item In the case of LTI system the Kalman filter itself becomes time invariant $\rightarrow$ constant Kalman gain.\\
 		ARE:
 		\begin{equation}
 			\begin{split}\label{eq:ss_kalman_udledning4}
 				&\Pi = {A} (\Pi^{-1} + {C}^T {R}^{-1} {C})^{-1} {A}^T + {Q}\\
 				& K = \Pi {C}^T ({C} \Pi {C}^T + {R})^{-1}
 			\end{split}
 		\end{equation}
 		\item Kalman filter is also very interesting from a leakage detection POV.
 	\end{itemize}
\end{frame}

%%%%%%%%%%%%%%%%%

\begin{frame}{Disturbance Estimation}{Water Consumption Data}
	\begin{itemize}
		\item Data of consumption pattern over a 35 day period courtesy of CSK and Grundfos.  
	\end{itemize}
		 \begin{figure}[h!]
			\centering
			\includegraphics[width=0.6\textwidth]{Topics/KalmanEstimator/Graphics/ConsumptionPattern.pdf}
			\caption{Consumption pattern over two days}
			\label{fig:Consumption_Pattern}
		\end{figure}
\end{frame}

%%%%%%%%%%%%%%%%%
	
\begin{frame}{Disturbance Estimation}{FFT of Consumption Pattern}
	\begin{itemize}
		\item Frequency analysis of the data. 
	\end{itemize}
	
	 \begin{figure}[h!]
		\centering
		\includegraphics[width=0.6\textwidth]{Topics/KalmanEstimator/Graphics/FFT.pdf}
		\caption{Amplitude and phase content of full consumption pattern}
		\label{fig:FFT_Consumption_Patter}
	\end{figure}
	
 We want a sparse representation. Largest frequency components are: DC, $24.05 \text{hr}$, $12.03 \text{hr}$, $8.02 \text{hr}$ and $5.99 \text{hr}$.

\end{frame}

%%%%%%%%%%%%%%%%%
\begin{frame}{Disturbance Estimation}{Approximation of Consumption}
		
	\begin{itemize}
		\item Fourth-order Fourier approximation of consumer demand:
	\end{itemize}
		
	\begin{equation} \label{eq:4th_order_approx}
		\begin{split}
			d_c(t) \approx& k_0 + k_1 cos(\omega_1 t + \phi_1) + k_2 cos(\omega_2 t + \phi_2)\\
			&+ k_3 cos(3\omega_3 t + \phi_3) + k_4 cos(4\omega_4 t + \phi_4)
		\end{split}
	\end{equation}

	\begin{itemize}
		\item We wish to model equation \ref{eq:4th_order_approx} as a state-space model:
	\end{itemize}
		
		\begin{equation*}
		\begin{split}
			\dot{x}=Ax\\
			y=Cx
		\end{split}
		\end{equation*}	
\end{frame}

%%%%%%%%%%%%%%%%%%%%%%%

\begin{frame}{Disturbance Estimation}{State-space Representation}
	\begin{itemize}
		\item Need to represent the evolution of the "states" in the approximation as a linear combination of states.
		\item This can not be achieved using the current states.   
	\end{itemize}
	
\begin{equation} \label{eq:consump_A}
	\dot{x} = 
	\begin{bmatrix}
		0 & 0 & 0 & 0 & 0 \\
		0 & 0 & -\omega_1 & 0 & 0 \\
		0 & \omega_1 & 0 & 0 & 0 \\
		0 & 0 & 0 & 0 & -\omega_2 \\
		0 & 0 & 0 & \omega_2 & 0 
	\end{bmatrix}
	\begin{bmatrix}
		k_0 \\
		k_1 cos(\omega_1 t) \\
		k_1 sin(\omega_1 t) \\
		k_2 cos(\omega_2 t) \\
		k_2 sin(\omega_2 t) 
	\end{bmatrix}
\end{equation}

\begin{equation}
	C = \begin{bmatrix} 1 & 1 & 0 & 1 & 0 \end{bmatrix}, \quad 
	\dot{x} = \begin{bmatrix}0 \\ -k_1\omega_1sin(\omega_1 t) \\ k_1\omega_1 cos(\omega_1 t) \\ -k_2\omega_2sin(\omega_2 t) \\ k_2\omega_2 cos(\omega_2 t)  \end{bmatrix} 
\end{equation}

	\begin{itemize}
	\item Fourth order doesn't fit the page!
\end{itemize}
\end{frame}

%%%%%%%%%%%%%%%%%%%%%%%

\begin{frame}{Disturbance Estimation}{Model vs. Data}
	\begin{itemize}
			\item The model compared to the real data - shown over 2 days. 
	\end{itemize}
		\begin{figure}[h!]
			\centering
			\includegraphics[width=0.6\textwidth]{Topics/KalmanEstimator/Graphics/Comparisson.pdf}
			\caption{Comparison of raw historical data, and model}
			\label{fig:Comparison}
		\end{figure}
	\begin{itemize}
		\item The model follows the visual trend in real data.
	\end{itemize}	
\end{frame}

%%%%%%%%%%%%%%%%%%%%%%%

\begin{frame}{Disturbance Estimation}{Kalman Filter Design}
	\textbf{Considerations when designing the KF.}
	\begin{itemize}
		\item In practice the Kalman gain is found using lqr in matlab.
		\item Stiffness of the filter is decided by Q-R ratio.
		\item Big R means uncertainty concerning observation and that we trust model much $\rightarrow$ stiff filter.
	\end{itemize}
\end{frame}
%
%\section{Network effects}
%\input{Topics/XX/XX}
%
%\section{Results}
%\input{Topics/XX/XX}















% ======================================================================




\section{References}
\begin{frame}{References}
	\bibliographystyle{ieeetran}
	\bibliography{../RefLib/CA7Projekt.bib}
\end{frame}

{\aauwavesbg
\begin{frame}[plain,noframenumbering]
  \finalpage{Open for questions!}
\end{frame}}
%%%%%%%%%%%%%%%%

\end{document}
