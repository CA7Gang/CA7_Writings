\begin{frame}{Disturbance Estimation}
	 \textbf{Objectives with the use of disturbance estimation}
	 \begin{itemize}
	 	\item VF-LQR controller needs an estimate of the consumer demand.
	 	\item The optimal estimator is the \textbf{Kalman Filter}.
	 \end{itemize}
\end{frame}

%%%%%%%%%%%%%%%%%

\begin{frame}{The Kalman filter}
 	\begin{itemize}
 		\item Need a linear model of consumer for the Kalman Filter
 		\item Under normal circumstances the Kalman gain is found recursively.
 		\item In the case of linear time invariant system, with covariances assumed to be constant over time, the Kalman Filter itself becomes time invariant $\rightarrow$ constant Kalman gain.\\
 		ARE:
 		\begin{equation}
 			\begin{split}\label{eq:ss_kalman_udledning4}
 				&\Pi = {A} (\Pi^{-1} + {C}^T {R}^{-1} {C})^{-1} {A}^T + {Q}\\
 				& K = \Pi {C}^T ({C} \Pi {C}^T + {R})^{-1}
 			\end{split}
 		\end{equation}
 		\item Kalman filter is also very interesting from a leakage detection POV.
 	\end{itemize}
\end{frame}

%%%%%%%%%%%%%%%%%

\begin{frame}{Water consumption data}
	\begin{itemize}
		\item Data of consumption pattern over a 35 day period obtained by CSK.  
	\end{itemize}
		 \begin{figure}[h!]
			\centering
			\includegraphics[width=0.6\textwidth]{Topics/KalmanEstimator/Graphics/ConsumptionPattern.pdf}
			\caption{Consumption pattern over two days}
			\label{fig:Consumption_Pattern}
		\end{figure}
\end{frame}

%%%%%%%%%%%%%%%%%
	
\begin{frame}{FFT of consumption pattern}
	\begin{itemize}
		\item Frequency analysis of the data. 
	\end{itemize}
	
	 \begin{figure}[h!]
		\centering
		\includegraphics[width=0.6\textwidth]{Topics/KalmanEstimator/Graphics/FFT.pdf}
		\caption{Consumption pattern over two days}
		\label{fig:FFT_Consumption_Patter}
	\end{figure}
	
	\begin{itemize}
		\item Highest frequency contents at periods: DC, 24.05h, 12.03h, 8.02h and 5.99h. 
	\end{itemize}
\end{frame}

%%%%%%%%%%%%%%%%%
\begin{frame}{Approximation of consumption}
		
	\begin{itemize}
		\item Fourth order Fourier approximation of consumer demand:
	\end{itemize}
		
	\begin{equation} \label{eq:4th_order_approx}
		\begin{split}
			d_c(t) \approx& k_0 + k_1 cos(\omega_1 t + \phi_1) + k_2 cos(\omega_2 t + \phi_2)\\
			&+ k_3 cos(3\omega_3 t + \phi_3) + k_4 cos(4\omega_4 t + \phi_4)
		\end{split}
	\end{equation}

	\begin{itemize}
		\item We wish to model \ref{eq:4th_order_approx} as a state space model:
	\end{itemize}
		
		\begin{equation*}
		\begin{split}
			\dot{x}=Ax\\
			y=Cx
		\end{split}
		\end{equation*}	
\end{frame}

%%%%%%%%%%%%%%%%%%%%%%%

\begin{frame}{State space representation}
	\begin{itemize}
		\item Need to represent the evolution of the "states" in the approximation as a linear combination of states.
		\item This can not be achieved using the current states.   
	\end{itemize}
	
\begin{equation} \label{eq:consump_A}
	\dot{x} = 
	\begin{bmatrix}
		0 & 0 & 0 & 0 & 0 \\
		0 & 0 & -\omega_1 & 0 & 0 \\
		0 & \omega_1 & 0 & 0 & 0 \\
		0 & 0 & 0 & 0 & -\omega_2 \\
		0 & 0 & 0 & \omega_2 & 0 
	\end{bmatrix}
	\begin{bmatrix}
		k_0 \\
		k_1 cos(\omega_1 t) \\
		k_1 sin(\omega_1 t) \\
		k_2 cos(\omega_2 t) \\
		k_2 sin(\omega_2 t) 
	\end{bmatrix}
\end{equation}

\begin{equation}
	y = \begin{bmatrix} 1 & 1 & 0 & 1 & 0 \end{bmatrix} 
	\begin{bmatrix}
		k_0 \\
		k_1 cos(\omega_1 t) \\
		k_1 sin(\omega_1 t) \\
		k_2 cos(\omega_2 t) \\
		k_2 sin(\omega_2 t) 
	\end{bmatrix}
\end{equation}

	\begin{itemize}
	\item Fourth order doesn't fit the page..
\end{itemize}
\end{frame}

%%%%%%%%%%%%%%%%%%%%%%%

\begin{frame}{Model vs. data}
	\begin{itemize}
			\item The model compared to the real data - shown over 2 days. 
	\end{itemize}
		\begin{figure}[h!]
			\centering
			\includegraphics[width=0.6\textwidth]{Topics/KalmanEstimator/Graphics/Comparisson.pdf}
			\caption{Comparison of raw historical data, and model}
			\label{fig:Comparison}
		\end{figure}
	\begin{itemize}
		\item The model follows the visual trend in real data.
	\end{itemize}	
\end{frame}

%%%%%%%%%%%%%%%%%%%%%%%

\begin{frame}{The Kalman Filter}
	\textbf{Considerations when designing the KF.}
	\begin{itemize}
		\item in practice the Kalman gain is found using lqr in matlab.
		\item Stiffness of the filter is decided by Q-R ratio.
		\item Big R means uncertainty concerning observation and that we trust model much $\rightarrow$ stiff filter.
	\end{itemize}
\end{frame}