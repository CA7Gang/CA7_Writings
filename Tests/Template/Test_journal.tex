
\documentclass[reportMaster.tex]{subfiles}
\begin{document}

\chapter{Test Journal} \label{app:tj_1}

Denne skabelon er tiltængt som lab-journal til undersøgende arbejde og er ikke tiltænkt til accepttest/modultest.\\
\textbf{Executed by:} \\
\textbf{Date:}

\tocless\section{Objective}
Hvad undersøges? Formålet med testen. “Relationen mellem x og y” “At måle kurveformer for spændingerne” “Vandflowet størrelse” “Hvordan UART protokollen fra GPS modulet ser ud” 


\tocless\section{Background}
Hvorfor undersøges dette? Begrund testens relevans. Referer til teori, analyse, eksterne kilder, kravspec. eller til en påstand. Her uddybes Objective.


Husk at hvis der refereres til kravsspecifikation. - Kravet skal fremgå her på skrift.

\tocless\section{Test subject}
Hvad testes der på? Indsæt diagrammer eller henvis til diagrammer i rapporten. Indsæt komponentdiagram eller systemdiagram fra datablad. Eksempler kunne være intern opbygning af en MOSFET eller et design af et filter der selv er designet. Diagrammerne er som udgangspunkt uden testopstilling, dvs. den ‘nøgne’ komponent/system uden forbindelser. 


\tocless\section{Equipment used}
Alt væsentligt udstyr skal beskrives entydigt. Inkluder evt. tabel
Hvilket udstyr er brugt til testen? Alt væsentligt udstyr skal beskrives entydigt.

\tocless\section{Test setup}
Hvordan er udstyret sat op? Oftest vises her en tegning over måleopstillingen, så man klart kan se, hvordan udstyret er tilsluttet. Noter filnavne hvis der er brugt SW eller simuleringsværktøj.

[XXX Indsæt tegning]


\tocless\section{Test procedure}
Hvordan udføres testen? Her beskrives klart og entydigt, hvordan målingen er foretaget inkl. alle ikke indlysende indstillinger af apparater. D.v.s. en fagmand M/K skal være i stand til at gentage målingen.


\tocless\section{Expected results}
Husk at definere hvad der forventes af resultater inden resultater præsenteres, således at der er noget at sætte resultaterne op imod


\tocless\section{Results and Comments}
Hvad viser testen? Nogle resultater kan med fordel flyttes til rapporten. Ofte angives tabeller i målejournalen og grafer i rapporten. Datafiler bør desuden vedlægges rapporten i bilag. 




\tocless\section{Sources of error and insecurities}
Her angives væsentlige fejlkilder og usikkerheder i.f.b. med målingen. Hvis der er uoverensstemmelser mellem beregnede/simulerede og målte data, må det forklares i rapporten. Her kan man evt. henvise til usikkerheder beskrevet i målejournalen.

Sammenligning af beregnede/simulerede og målte resultater (konklusionen) er vigtig, men skal beskrives i rapporten. Denne sag er målejournalen uvedkommend.

Du må godt skrive din konklusion ind her for nu, men den skal flyttes over i rapporten! 

XXX
 


% \renewcommand\bibname{Referencer}
% \bibliographystyle{abbrv}
% \bibliography{Biblo}  
% \bibliographystyleA{abbrv}
% \bibliographyA{Biblo}  

\end{document}